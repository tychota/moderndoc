% =============================================================================
% REPORT TEMPLATE — Technical/Business Report
% =============================================================================
% Compile with: lualatex --shell-escape report.tex
%               biber report
%               lualatex --shell-escape report.tex (x2)
% =============================================================================

% --- PDF 2.0 with PDF/A-4 archival format (must be before \documentclass) ---
% PDF/A-4 (ISO 19005-4:2020) is the archival standard based on PDF 2.0
\DocumentMetadata{
  pdfversion  = 2.0,
  pdfstandard = a-4,
  lang        = en-US,
}

% Suppress expected warning from latex-lab tagging (modifies \@footnotemark)
\RequirePackage{silence}
\WarningFilter{latex}{Command \@footnotemark}

\documentclass[
  11pt,
  a4paper,
  oneside,
  headings      = normal,
  numbers       = noenddot,
  listof        = totoc,
  bibliography  = totoc,
]{scrreprt}

% =============================================================================
% PREAMBLE
% =============================================================================
\usepackage[report, citestyle=numeric]{moderndoc}

% --- Bibliography file ---
\addbibresource{references.bib}

% --- Chapter styling ---
\addtokomafont{chapter}{\headingfont}
\addtokomafont{chapterprefix}{\headingfont\Large}
\renewcommand*{\raggedchapter}{\raggedleft}
\renewcommand*{\chapterformat}{%
  \mbox{\scalebox{2}{\color{mdocsecondary}\thechapter}}%
}
\RedeclareSectionCommand[
  beforeskip = 0pt,
  afterskip  = 2cm,
  innerskip  = 0.8cm,
]{chapter}

% --- Document metadata ---
\hypersetup{
  pdftitle    = {Technical Report},
  pdfauthor   = {Author Name},
  pdfsubject  = {Technical Analysis},
  pdfkeywords = {report, analysis, technical},
}

% =============================================================================
% REPORT INFORMATION
% =============================================================================
\newcommand{\reporttitle}{Technical Analysis Report}
\newcommand{\reportsubtitle}{A Comprehensive Review of System Performance}
\newcommand{\reportauthor}{Technical Team}
\newcommand{\reportorg}{Organization Name}
\newcommand{\reportnumber}{TR-2025-001}
\newcommand{\reportversion}{1.0}
\newcommand{\reportdate}{\today}

% =============================================================================
% DOCUMENT CONTENT
% =============================================================================
\begin{document}

% =============================================================================
% TITLE PAGE
% =============================================================================
\begin{titlepage}
  \centering

  % Organization logo placeholder
  \vspace*{1cm}
  {\headingfont\Large\reportorg\par}

  \vspace{3cm}

  % Report type
  {\headingfont\large Technical Report\par}
  \vspace{0.5cm}
  {\small Report Number: \reportnumber\par}

  \vspace{2cm}

  % Title
  {\headingfont\Huge\bfseries\reporttitle\par}
  \vspace{0.8cm}
  {\headingfont\Large\reportsubtitle\par}

  \vspace{2cm}

  % Author
  {\Large Prepared by:\par}
  \vspace{0.3cm}
  {\Large\reportauthor\par}

  \vfill

  % Footer information
  \begin{tabular}{ll}
    \textbf{Version:} & \reportversion \\
    \textbf{Date:} & \reportdate \\
    \textbf{Classification:} & Internal Use Only \\
  \end{tabular}

\end{titlepage}

% =============================================================================
% FRONT MATTER
% =============================================================================
\pagenumbering{roman}
\pagestyle{plain}

% --- Document Control ---
\chapter*{Document Control}
\addcontentsline{toc}{chapter}{Document Control}

\section*{Revision History}

\begin{table}[h]
  \begin{tblr}{
    colspec = {llXl},
    row{1}  = {font=\bfseries\sffamily, bg=gray!15},
    hlines  = {0.5pt, gray!50},
    rowsep  = 3pt,
  }
    Version & Date       & Changes                    & Author \\
    1.0     & 2025-01-15 & Initial release            & Technical Team \\
    0.9     & 2025-01-10 & Draft for internal review  & Technical Team \\
    0.5     & 2025-01-05 & Initial draft              & Technical Team \\
  \end{tblr}
\end{table}

\section*{Distribution List}

\begin{itemize}
  \item Executive Leadership Team
  \item Technical Advisory Board
  \item Project Stakeholders
  \item Quality Assurance Team
\end{itemize}

\section*{Approval}

\begin{table}[h]
  \begin{tblr}{
    colspec = {lll},
    row{1}  = {font=\bfseries\sffamily, bg=gray!15},
    hlines  = {0.5pt, gray!50},
    rowsep  = 3pt,
  }
    Role              & Name          & Date \\
    Author            & Jane Doe      & 2025-01-15 \\
    Reviewer          & John Smith    & 2025-01-14 \\
    Approver          & Alice Johnson & 2025-01-15 \\
  \end{tblr}
\end{table}

% --- Executive Summary ---
\chapter*{Executive Summary}
\addcontentsline{toc}{chapter}{Executive Summary}

This report presents a comprehensive analysis of the system performance over
the past quarter. The key findings and recommendations are summarized below.

\section*{Key Findings}

\begin{enumerate}
  \item \textbf{Performance Improvement:} System throughput increased by 25\%
    following the optimization efforts implemented in Q3.

  \item \textbf{Reliability Metrics:} Uptime exceeded 99.9\%, meeting the
    service level agreement targets.

  \item \textbf{Cost Reduction:} Infrastructure costs were reduced by 15\%
    through resource optimization and consolidation.

  \item \textbf{Security Posture:} Zero critical security incidents were
    recorded during the reporting period.
\end{enumerate}

\section*{Recommendations}

\begin{itemize}
  \item Continue investment in automation and monitoring capabilities
  \item Expand capacity planning to accommodate projected growth
  \item Implement additional redundancy measures for critical components
  \item Enhance documentation and knowledge transfer processes
\end{itemize}

% --- Table of Contents ---
\tableofcontents

% --- List of Figures and Tables ---
\listoffigures
\listoftables

% =============================================================================
% MAIN MATTER
% =============================================================================
\pagenumbering{arabic}
\pagestyle{mdocreport}

% =====================================
\chapter{Introduction}
\label{ch:intro}
% =====================================

\section{Purpose}

This report documents the technical analysis conducted during the fourth
quarter of 2024. The primary objectives of this analysis were to:

\begin{itemize}
  \item Evaluate current system performance against established baselines
  \item Identify areas requiring improvement or optimization
  \item Provide actionable recommendations for future enhancements
  \item Document lessons learned and best practices
\end{itemize}

\section{Scope}

The analysis covers the following system components and processes:

\begin{description}
  \item[Infrastructure] Server hardware, network equipment, and storage systems
  \item[Applications] Core business applications and supporting services
  \item[Data] Database systems, data pipelines, and analytics platforms
  \item[Security] Access controls, monitoring, and incident response
\end{description}

\section{Methodology}

The analysis employed a structured approach combining quantitative metrics
with qualitative assessments:

\begin{citedquote}{W.\ Edwards Deming}{1986}
In God we trust; all others must bring data.
\end{citedquote}

Our methodology followed the Plan-Do-Check-Act (PDCA) cycle, with particular
emphasis on data-driven decision making.

% =====================================
\chapter{Technical Analysis}
\label{ch:analysis}
% =====================================

\section{Performance Metrics}

\subsection{System Throughput}

Table~\ref{tab:throughput} presents the monthly throughput metrics for the
reporting period.

\begin{table}[htbp]
  \centering
  \caption{Monthly system throughput (transactions per second)}
  \label{tab:throughput}
  \begin{tblr}{
    colspec = {lrrr},
    row{1}  = {font=\bfseries\sffamily, bg=gray!15},
    hlines  = {0.5pt, gray!50},
    rowsep  = 3pt,
  }
    Month     & Average TPS & Peak TPS & 99th Percentile \\
    October   & 1,234       & 2,456    & 1,890 \\
    November  & 1,345       & 2,678    & 2,012 \\
    December  & 1,456       & 2,891    & 2,234 \\
  \end{tblr}
\end{table}

\begin{notebox}[Performance Improvement]
The 18\% improvement in average throughput from October to December reflects
the successful implementation of caching optimizations deployed in early
November.
\end{notebox}

\subsection{Response Times}

Response time analysis revealed consistent performance across all service tiers:

\begin{itemize}
  \item \textbf{Tier 1 (Critical):} Average 45ms, 99th percentile 120ms
  \item \textbf{Tier 2 (Standard):} Average 150ms, 99th percentile 450ms
  \item \textbf{Tier 3 (Background):} Average 500ms, 99th percentile 1500ms
\end{itemize}

\section{Reliability Analysis}

\subsection{Availability Metrics}

System availability exceeded targets for all measured components:

\begin{equation}
  \text{Availability} = \frac{\text{Uptime}}{\text{Total Time}} \times 100\%
  = \frac{8,755.2}{8,760} \times 100\% = 99.945\%
  \label{eq:availability}
\end{equation}

As shown in \cref{eq:availability}, the achieved availability of 99.945\%
exceeded the SLA target of 99.9\%.

\subsection{Incident Analysis}

A total of 12 incidents were recorded during the quarter:

\begin{table}[htbp]
  \centering
  \caption{Incident classification by severity}
  \label{tab:incidents}
  \begin{tblr}{
    colspec = {lccl},
    row{1}  = {font=\bfseries\sffamily, bg=gray!15},
    hlines  = {0.5pt, gray!50},
    rowsep  = 3pt,
  }
    Severity & Count & MTTR (hours) & Primary Cause \\
    Critical & 0     & N/A          & --- \\
    High     & 2     & 1.5          & Configuration drift \\
    Medium   & 5     & 4.2          & Resource exhaustion \\
    Low      & 5     & 8.0          & Scheduled maintenance \\
  \end{tblr}
\end{table}

\begin{warningbox}
While no critical incidents occurred, the two high-severity incidents related
to configuration drift highlight the need for improved configuration
management practices.
\end{warningbox}

% =====================================
\chapter{Recommendations}
\label{ch:recommendations}
% =====================================

Based on the analysis presented in \cref{ch:analysis}, the following
recommendations are proposed:

\section{Short-Term Actions (0--3 months)}

\begin{enumerate}
  \item \textbf{Configuration Management Enhancement}
    \begin{itemize}
      \item Implement infrastructure-as-code for all production systems
      \item Establish automated configuration drift detection
      \item Deploy change approval workflows for production modifications
    \end{itemize}

  \item \textbf{Monitoring Improvements}
    \begin{itemize}
      \item Expand metric collection to cover all critical paths
      \item Implement predictive alerting based on trend analysis
      \item Create executive dashboards for key performance indicators
    \end{itemize}
\end{enumerate}

\section{Medium-Term Actions (3--6 months)}

\begin{enumerate}
  \item \textbf{Capacity Planning}
    \begin{itemize}
      \item Develop growth projections based on business forecasts
      \item Identify scaling bottlenecks and develop mitigation plans
      \item Establish resource reservation procedures for peak periods
    \end{itemize}

  \item \textbf{Disaster Recovery Enhancement}
    \begin{itemize}
      \item Conduct full-scale disaster recovery exercises
      \item Implement cross-region replication for critical data
      \item Update recovery time objectives based on business requirements
    \end{itemize}
\end{enumerate}

\section{Long-Term Actions (6--12 months)}

\begin{enumerate}
  \item \textbf{Architecture Modernization}
    \begin{itemize}
      \item Evaluate migration to cloud-native architecture
      \item Assess microservices decomposition opportunities
      \item Plan technology refresh for aging infrastructure components
    \end{itemize}
\end{enumerate}

% =====================================
\chapter{Conclusion}
\label{ch:conclusion}
% =====================================

This report has presented a comprehensive analysis of system performance for
Q4 2024. The key conclusions are:

\begin{enumerate}
  \item Overall system performance meets or exceeds established targets
  \item Infrastructure investments have yielded measurable improvements
  \item Identified areas for improvement have clear remediation paths
  \item The organization is well-positioned for anticipated growth
\end{enumerate}

The recommendations outlined in \cref{ch:recommendations} should be
prioritized and incorporated into the operational planning process for the
upcoming fiscal year.

% =============================================================================
% BACK MATTER
% =============================================================================

% --- Bibliography ---
\printbibliography

% --- Appendices ---
\appendix

\chapter{Detailed Metrics}
\label{app:metrics}

This appendix contains detailed metric data supporting the analysis in the
main report. Raw data is available upon request from the Technical Team.

\section{Daily Throughput Data}

Detailed daily throughput measurements are maintained in the monitoring
system and can be accessed via the operations dashboard.

\section{Incident Timeline}

Complete incident timelines with root cause analysis are documented in the
incident management system.

\chapter{Glossary}
\label{app:glossary}

\begin{description}
  \item[MTTR] Mean Time to Recovery --- the average time required to restore
    service after an incident.
  \item[SLA] Service Level Agreement --- contractual commitments for service
    availability and performance.
  \item[TPS] Transactions Per Second --- a measure of system throughput.
  \item[Uptime] The total time a system is operational and available.
\end{description}

\end{document}
