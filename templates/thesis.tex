% =============================================================================
% THESIS TEMPLATE — Academic Thesis/Dissertation
% =============================================================================
% Compile with: lualatex --shell-escape thesis.tex
%               biber thesis
%               lualatex --shell-escape thesis.tex (x2)
% =============================================================================

% --- PDF 2.0 with PDF/A-4 archival format (must be before \documentclass) ---
% PDF/A-4 (ISO 19005-4:2020) is the archival standard based on PDF 2.0
\DocumentMetadata{
  pdfversion  = 2.0,
  pdfstandard = a-4,
  lang        = en-US,
}

% Suppress expected warning from latex-lab tagging (modifies \@footnotemark)
\RequirePackage{silence}
\WarningFilter{latex}{Command \@footnotemark}

\documentclass[
  11pt,
  a4paper,
  twoside,
  openright,
  cleardoublepage = empty,  % Use KOMA-Script option instead of emptypage package
  chapterprefix = true,
  numbers       = noenddot,
  headings      = normal,
]{scrbook}

% =============================================================================
% PREAMBLE
% =============================================================================
\usepackage[thesis, citestyle=authoryear]{moderndoc}

% --- Bibliography file ---
\addbibresource{references.bib}

% --- Thesis-specific packages ---
\usepackage{setspace}

% --- Chapter styling ---
\addtokomafont{chapter}{\headingfont}
\addtokomafont{chapterprefix}{\headingfont\Large}
\renewcommand*{\raggedchapter}{\raggedleft}
\renewcommand*{\chapterformat}{%
  \mbox{\chapappifchapterprefix{\nobreakspace}\thechapter\autodot}%
}
\RedeclareSectionCommand[
  beforeskip = 0pt,
  afterskip  = 2cm,
]{chapter}

% --- Header/footer adjustments for thesis (scrlayer-scrpage) ---
\defpagestyle{thesis}{%
  % head: (even, center, odd)
  {\thepage\hfill\textit{\leftmark}}{}{\textit{\rightmark}\hfill\thepage}%
}{%
  % foot: (even, center, odd)
  {}{}{}%
}
\KOMAoptions{headsepline=0.4pt}

% --- Front matter style ---
\newcommand{\frontmatterstyle}{%
  \pagestyle{plain}
  \pagenumbering{roman}
}

% --- Main matter style ---
\newcommand{\mainmatterstyle}{%
  \pagestyle{thesis}
  \pagenumbering{arabic}
}

% --- Document metadata ---
\hypersetup{
  pdftitle    = {Your Thesis Title},
  pdfauthor   = {Your Name},
  pdfsubject  = {Doctoral Dissertation / Master's Thesis},
  pdfkeywords = {keyword1, keyword2, keyword3},
}

% =============================================================================
% THESIS INFORMATION
% =============================================================================
\newcommand{\thesistitle}{A Comprehensive Study of Something Important:\\
  With a Detailed Subtitle That Explains the Focus}
\newcommand{\theauthor}{Your Full Name}
\newcommand{\thesistype}{Doctoral Dissertation}  % or Master's Thesis
\newcommand{\department}{Department of Computer Science}
\newcommand{\university}{University of Example}
\newcommand{\submissiondate}{December 2025}
\newcommand{\supervisor}{Prof.\ Jane Doe}
\newcommand{\cosupervisor}{Dr.\ John Smith}

% =============================================================================
% DOCUMENT CONTENT
% =============================================================================
\begin{document}

% =============================================================================
% FRONT MATTER
% =============================================================================
\frontmatter
\frontmatterstyle

% --- Title Page ---
\begin{titlepage}
  \centering
  \vspace*{2cm}
  
  {\headingfont\Large\university\par}
  \vspace{0.5cm}
  {\headingfont\large\department\par}
  
  \vspace{3cm}
  
  {\headingfont\LARGE\bfseries\thesistitle\par}
  
  \vspace{2cm}
  
  {\Large\thesistype\par}
  
  \vspace{2cm}
  
  {\Large\theauthor\par}
  
  \vfill
  
  {\large Submitted in partial fulfillment of the requirements\\
  for the degree of Doctor of Philosophy\par}
  
  \vspace{1cm}
  
  {\large\submissiondate\par}
\end{titlepage}

% --- Copyright Page ---
\thispagestyle{empty}
\null
\vfill
\begin{center}
  © \the\year\ \theauthor\\[1em]
  All rights reserved.
\end{center}
\clearpage

% --- Abstract ---
\begin{thesisabstract}
This thesis investigates [main topic]. The central research question is:
[research question].

The first contribution of this work is [contribution 1]. We demonstrate that
[finding 1].

The second contribution is [contribution 2]. Through [methodology], we show
that [finding 2].

Our third contribution provides [contribution 3]. Experiments on [benchmarks]
demonstrate [results].

The implications of this research extend to [broader impact]. Future directions
include [future work].

\textbf{Keywords:} keyword1, keyword2, keyword3, keyword4, keyword5
\end{thesisabstract}

% --- French Abstract (Résumé) ---
\begin{abstractfr}
Cette thèse étudie [sujet principal]. La question de recherche centrale est:
[question de recherche].

La première contribution de ce travail est [contribution 1]. Nous démontrons
que [résultat 1].

\textbf{Mots-clés:} mot-clé1, mot-clé2, mot-clé3, mot-clé4, mot-clé5
\end{abstractfr}

% --- Acknowledgements ---
\chapter*{Acknowledgements}
\addcontentsline{toc}{chapter}{Acknowledgements}

First and foremost, I would like to express my sincere gratitude to my
supervisor, \supervisor, for their invaluable guidance, support, and patience
throughout this journey.

I am also deeply thankful to my co-supervisor, \cosupervisor, for their
insights and feedback.

I would like to thank my colleagues in the [lab name] for the stimulating
discussions and collaborative environment.

Finally, I am grateful to my family and friends for their unconditional support
and encouragement.

% --- Table of Contents ---
\tableofcontents

% --- List of Figures ---
\listoffigures
\addcontentsline{toc}{chapter}{List of Figures}

% --- List of Tables ---
\listoftables
\addcontentsline{toc}{chapter}{List of Tables}

% =============================================================================
% MAIN MATTER
% =============================================================================
\mainmatter
\mainmatterstyle
\onehalfspacing

% =====================================
\chapter{Introduction}
\label{ch:introduction}
% =====================================

\section{Motivation}

The opening section establishes the broader context and importance of your
research area. Why does this problem matter? What are the practical or
theoretical implications?

\begin{citedquote}{Influential Researcher}{2015}
A relevant quote that captures the essence of the challenge or the vision that
motivates this research.
\end{citedquote}

\section{Problem Statement}

Clearly define the specific problem that this thesis addresses. What gap exists
in current knowledge or approaches?

The key challenges include:
\begin{enumerate}
  \item Challenge one: description of the first major challenge
  \item Challenge two: description of the second major challenge  
  \item Challenge three: description of the third major challenge
\end{enumerate}

\section{Research Questions}

This thesis seeks to answer the following research questions:

\begin{description}
  \item[RQ1:] First research question?
  \item[RQ2:] Second research question?
  \item[RQ3:] Third research question?
\end{description}

\section{Contributions}

The main contributions of this thesis are:

\begin{enumerate}
  \item \textbf{Contribution 1.} Description of the first contribution and its
    significance.
  \item \textbf{Contribution 2.} Description of the second contribution.
  \item \textbf{Contribution 3.} Description of the third contribution.
\end{enumerate}

\section{Thesis Structure}

The remainder of this thesis is organized as follows:

\textbf{\Cref{ch:background}} provides the necessary background on [topics].

\textbf{\Cref{ch:relatedwork}} surveys related work in [areas].

\textbf{\Cref{ch:methodology}} presents the proposed methodology.

\textbf{\Cref{ch:evaluation}} describes the experimental evaluation.

\textbf{\Cref{ch:conclusion}} concludes the thesis and discusses future work.

% =====================================
\chapter{Background}
\label{ch:background}
% =====================================

This chapter provides the theoretical foundations and technical background
necessary to understand the contributions of this thesis.

\section{Foundational Concepts}

\subsection{Concept One}

Explanation of the first foundational concept. Include definitions, notation,
and examples as appropriate.

\begin{notebox}[Definition]
A \concept{key term} is defined as follows: [formal definition].
\end{notebox}

\subsection{Concept Two}

Explanation of the second foundational concept.

\section{Technical Background}

\subsection{Algorithms and Methods}

Description of relevant algorithms. Include pseudocode where appropriate:

\begin{minted}{python}
def algorithm_name(input_data, parameters):
    """
    Brief description of what this algorithm does.
    
    Args:
        input_data: Description of input
        parameters: Configuration parameters
    
    Returns:
        Output description
    """
    # Step 1: Initialize
    result = initialize(parameters)
    
    # Step 2: Main loop
    for item in input_data:
        result = process(result, item)
    
    # Step 3: Finalize
    return finalize(result)
\end{minted}

\section{Summary}

Brief summary of the background material and how it relates to the thesis
contributions.

% =====================================
\chapter{Related Work}
\label{ch:relatedwork}
% =====================================

This chapter surveys the existing literature relevant to this thesis.

\section{First Area of Related Work}

Discussion of the first category of related work. Compare approaches, identify
strengths and limitations.

\subsection{Early Approaches}

Historical perspective on early work in this area.

\subsection{Recent Advances}

More recent developments and the current state of the art.

\section{Second Area of Related Work}

Discussion of another relevant area.

\section{Positioning of This Thesis}

How does this thesis relate to and extend existing work? What gaps does it
address?

% =====================================
\chapter{Methodology}
\label{ch:methodology}
% =====================================

This chapter presents the proposed methodology in detail.

\section{Overview}

High-level description of the approach.

\section{Component One}

Detailed description of the first component.

\section{Component Two}

Detailed description of the second component.

\section{Theoretical Analysis}

If applicable, provide theoretical analysis, proofs, or formal properties.

\begin{notebox}[Theorem 1]
Statement of the theorem.
\end{notebox}

\textit{Proof.} Sketch of the proof or reference to appendix for full details.
\hfill$\mdlgwhtsquare$

% =====================================
\chapter{Experimental Evaluation}
\label{ch:evaluation}
% =====================================

This chapter presents the experimental evaluation of the proposed approach.

\section{Experimental Setup}

\subsection{Datasets}

Description of the datasets used for evaluation.

\begin{table}[htbp]
  \centering
  \caption{Dataset statistics}
  \label{tab:datasets}
  \begin{tblr}{
    colspec = {lrrrl},
    row{1}  = {font=\bfseries\sffamily, bg=gray!15},
    hlines  = {0.5pt, gray!50},
    rowsep  = 3pt,
  }
    Dataset   & Samples & Features & Classes & Source \\
    Dataset A & 10,000  & 256      & 10      & [source] \\
    Dataset B & 50,000  & 512      & 100     & [source] \\
    Dataset C & 100,000 & 1,024    & 1,000   & [source] \\
  \end{tblr}
\end{table}

\subsection{Baselines}

Description of baseline methods for comparison.

\subsection{Evaluation Metrics}

Description of the metrics used to evaluate performance.

\subsection{Implementation Details}

Technical details about the implementation.

\section{Results}

\subsection{Main Results}

Presentation and analysis of the main results.

\subsection{Ablation Studies}

Analysis of component contributions.

\subsection{Qualitative Analysis}

Qualitative examples and case studies.

\section{Discussion}

Interpretation of results, limitations, and implications.

% =====================================
\chapter{Conclusion}
\label{ch:conclusion}
% =====================================

\section{Summary of Contributions}

This thesis has made the following contributions:

\begin{enumerate}
  \item Summary of contribution 1
  \item Summary of contribution 2
  \item Summary of contribution 3
\end{enumerate}

\section{Answers to Research Questions}

\paragraph{RQ1:} Summary of findings for the first research question.

\paragraph{RQ2:} Summary of findings for the second research question.

\paragraph{RQ3:} Summary of findings for the third research question.

\section{Limitations}

Discussion of limitations and scope of the work.

\section{Future Work}

Directions for future research building on this thesis.

% =============================================================================
% BACK MATTER
% =============================================================================
\backmatter

% --- Bibliography ---
\printbibliography[heading=bibintoc]

% --- Appendices ---
\startappendices

\chapter{Additional Experiments}
\label{app:experiments}

Additional experimental results that supplement the main evaluation.

\chapter{Proofs}
\label{app:proofs}

Complete proofs of theorems stated in the main text.

\end{document}
