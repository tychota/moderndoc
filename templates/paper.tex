% =============================================================================
% PAPER TEMPLATE — Two-Column Conference Paper
% =============================================================================
% Compile with: lualatex --shell-escape paper.tex
%               biber paper
%               lualatex --shell-escape paper.tex
% =============================================================================

% --- PDF 2.0 and metadata (must be before \documentclass) ---
\DocumentMetadata{
  pdfversion = 2.0,
  lang       = en-US,
}

\documentclass[11pt, a4paper, twocolumn]{scrartcl}

% =============================================================================
% PREAMBLE
% =============================================================================
\usepackage[paper, citestyle=numeric]{moderndoc}

% --- Bibliography file ---
\addbibresource{references.bib}

% --- Additional paper-specific settings ---
\RedeclareSectionCommand[
  beforeskip = 1.5ex plus 0.5ex minus 0.2ex,
  afterskip  = 1ex plus 0.2ex,
]{section}

\RedeclareSectionCommand[
  beforeskip = 1.2ex plus 0.4ex minus 0.2ex,
  afterskip  = 0.8ex plus 0.2ex,
]{subsection}

% Tighter lists for papers
\setlist{
  topsep     = 0.3em,
  itemsep    = 0.15em,
  parsep     = 0pt,
  leftmargin = 1.2em,
}

% --- Document metadata ---
\hypersetup{
  pdftitle    = {Your Paper Title},
  pdfauthor   = {Author Names},
  pdfsubject  = {Conference Name 2025},
  pdfkeywords = {keyword1, keyword2, keyword3},
}

% =============================================================================
% TITLE CONFIGURATION
% =============================================================================
\title{A Long and Descriptive Paper Title\\That May Span Multiple Lines}
\author{%
  First Author$^{1}$ \quad
  Second Author$^{2}$ \quad
  Third Author$^{1,2}$\\[0.5em]
  \small$^{1}$University of Example, Department of Computer Science\\
  \small$^{2}$Research Institute, Laboratory Name\\[0.3em]
  \small\texttt{\{first, third\}@example.edu}, \texttt{second@research.org}
}
\date{}  % No date for conference papers

% =============================================================================
% DOCUMENT CONTENT
% =============================================================================
\begin{document}

\maketitle

% --- Abstract ---
\begin{abstract}
\noindent
This paper presents a novel approach to [problem domain]. We propose a method
that [key contribution 1] and [key contribution 2]. Our experiments on
[benchmark/dataset] demonstrate that our approach achieves [main result],
outperforming state-of-the-art methods by [margin]. We also provide
[additional contribution, e.g., theoretical analysis, open-source implementation].
Our findings suggest that [broader implication].
\end{abstract}

% =============================================================================
\section{Introduction}
% =============================================================================

The opening paragraph should establish the context and importance of the
research area. What is the broader problem domain? Why does it matter?

The second paragraph typically identifies the specific gap or challenge that
this paper addresses. What limitation exists in current approaches?

\begin{paperquote}
Prior work has focused primarily on X, leaving Y relatively unexplored despite
its practical importance in real-world applications.
\end{paperquote}

In this paper, we present:
\begin{itemize}
  \item A novel method for [task] that [key innovation]
  \item Theoretical analysis showing [property/guarantee]
  \item Comprehensive experiments demonstrating [improvement]
\end{itemize}

% =============================================================================
\section{Related Work}
% =============================================================================

\subsection{First Category of Related Work}

Discuss the first category of related work. Compare and contrast with your
approach. What are the limitations that your work addresses?

\subsection{Second Category of Related Work}

Another category of related work with similar structure.

% =============================================================================
\section{Methodology}
% =============================================================================

\subsection{Problem Formulation}

Let $\mathcal{X}$ denote the input space and $\mathcal{Y}$ the output space.
Given a dataset $\mathcal{D} = \{(x_i, y_i)\}_{i=1}^{n}$, our goal is to learn
a function $f: \mathcal{X} \to \mathcal{Y}$ that minimizes:

\begin{equation}
  \mathcal{L}(f) = \sum_{i=1}^{n} \ell(f(x_i), y_i) + \lambda \Omega(f)
  \label{eq:objective}
\end{equation}

where $\ell$ is the loss function and $\Omega$ is a regularization term.

\subsection{Proposed Approach}

Our method consists of three main components:

\paragraph{Component 1.}
Description of the first component of your method.

\paragraph{Component 2.}
Description of the second component.

\subsection{Algorithm}

The complete algorithm is presented below:

\begin{minted}[fontsize=\footnotesize, linenos]{python}
def our_algorithm(X, y, params):
    """Main algorithm implementation."""
    model = initialize_model(params)
    
    for epoch in range(params.epochs):
        # Training loop
        loss = compute_loss(model, X, y)
        gradients = compute_gradients(loss)
        model = update_model(model, gradients)
    
    return model
\end{minted}

% =============================================================================
\section{Experiments}
% =============================================================================

\subsection{Experimental Setup}

\paragraph{Datasets.}
We evaluate on three benchmark datasets: Dataset A (10K samples), Dataset B
(50K samples), and Dataset C (100K samples).

\paragraph{Baselines.}
We compare against: Method 1~\autocite{knuth1984texbook}, Method
2~\autocite{lamport1994latex}, and the state-of-the-art Method 3.

\paragraph{Metrics.}
We report accuracy, F1-score, and inference time.

\subsection{Main Results}

\begin{table}[t]
  \centering
  \caption{Performance comparison on benchmark datasets}
  \label{tab:results}
  \begin{tblr}{
    colspec = {lcccc},
    row{1}  = {font=\bfseries\footnotesize\sffamily, bg=gray!15},
    hlines  = {0.5pt, gray!50},
    rowsep  = 2pt,
    colsep  = 4pt,
    cell{1}{1} = {r=2}{},
    cell{1}{2} = {c=2}{c},
    cell{1}{4} = {c=2}{c},
  }
    Method & Dataset A & & Dataset B & \\
           & Acc. & F1  & Acc. & F1  \\
    Method 1 & 85.2 & 84.1 & 82.3 & 81.5 \\
    Method 2 & 87.4 & 86.2 & 84.6 & 83.8 \\
    Method 3 & 89.1 & 88.3 & 86.2 & 85.4 \\
    \textbf{Ours} & \textbf{91.3} & \textbf{90.5} & \textbf{88.7} & \textbf{87.9} \\
  \end{tblr}
\end{table}

As shown in \cref{tab:results}, our method achieves the best performance across
all datasets and metrics.

\subsection{Ablation Study}

To understand the contribution of each component, we conduct ablation
experiments by removing one component at a time.

% =============================================================================
\section{Discussion}
% =============================================================================

\paragraph{Limitations.}
While our method achieves strong results, it has some limitations. First, [limitation 1].
Second, [limitation 2].

\paragraph{Future Work.}
Several directions remain for future investigation: [direction 1] and
[direction 2].

% =============================================================================
\section{Conclusion}
% =============================================================================

We presented [method name], a novel approach for [task]. Our key contributions
include [contribution 1] and [contribution 2]. Experiments demonstrate
state-of-the-art performance on [benchmarks], with improvements of [X\%] over
previous methods. [Final sentence about impact or future directions.]

% =============================================================================
% ACKNOWLEDGEMENTS
% =============================================================================
\section*{Acknowledgements}

This work was supported by [funding source]. We thank [people] for helpful
discussions.

% =============================================================================
% BIBLIOGRAPHY
% =============================================================================
{\small
\printbibliography
}

\end{document}
