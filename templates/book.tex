% =============================================================================
% BOOK TEMPLATE — Technical/Academic Book
% =============================================================================
% Compile with: lualatex --shell-escape book.tex
%               biber book
%               lualatex --shell-escape book.tex (x2)
% =============================================================================

% --- PDF 2.0 with PDF/A-4 archival format (must be before \documentclass) ---
% PDF/A-4 (ISO 19005-4:2020) is the archival standard based on PDF 2.0
\DocumentMetadata{
  pdfversion=2.0,
  pdfstandard=a-4,
  lang=en-US,
}

\documentclass[
  book,
  11pt,
  a4paper,
  twoside,
  open=right,
  cleardoublepage=empty,  % Use KOMA-Script option instead of emptypage package
  chapterprefix=true,
  numbers=noenddot,
  headings=big,
  citestyle=authoryear,
  DIV=12,
  BCOR=10mm
]{modern-doc}

% =============================================================================
% PREAMBLE
% =============================================================================
% --- Bibliography file ---
\addbibresource{references.bib}

% --- Book-specific packages ---
\usepackage{setspace}
\usepackage{lettrine}    % Drop caps for chapter openings

% --- Part styling ---
\addtokomafont{part}{\headingfont\Huge}
\addtokomafont{partprefix}{\headingfont\Large}
\renewcommand*{\raggedpart}{\centering}

% --- Chapter styling (decorative) ---
\addtokomafont{chapter}{\headingfont}
\addtokomafont{chapterprefix}{\headingfont\Large\color{mdocGray}}

\renewcommand*{\raggedchapter}{\raggedleft}
\renewcommand*{\chapterformat}{%
  \mbox{\scalebox{2.5}{\color{mdocGray}\thechapter}}%
}

\RedeclareSectionCommand[
  beforeskip = -1sp,  % Removes vspace but keeps heading
  afterskip  = 3cm,
  innerskip  = 1cm,
]{chapter}

% --- Header/footer for books (scrlayer-scrpage) ---
\defpagestyle{book}{%
  % head: (even, center, odd)
  {\textit{\leftmark}}{}{\textit{\rightmark}}%
}{%
  % foot: (even, center, odd)
  {}{\thepage}{}%
}
\KOMAoptions{headsepline=0.4pt}
\pagestyle{book}

% --- Drop cap configuration ---
% IBM Plex Sans Bold pairs well with Plex Serif body text
\renewcommand{\LettrineFontHook}{\headingfont\bfseries\color{mdocGray}}
\setcounter{DefaultLines}{2}
\renewcommand{\DefaultLoversize}{0.15}
\renewcommand{\DefaultLraise}{0.1}
\setlength{\DefaultFindent}{3pt}
\setlength{\DefaultNindent}{0pt}

% Note: The chapterquote environment is now provided by modern-doc.sty
% Legacy alias for backward compatibility
\newcommand{\chapterepigraph}[2]{%
  \begin{chapterquote}{#2}{}
    #1
  \end{chapterquote}
}

% --- Document metadata ---
\hypersetup{
  pdftitle    = {Book Title},
  pdfauthor   = {Author Name},
  pdfsubject  = {Technical Book},
  pdfkeywords = {keyword1, keyword2, keyword3},
}

% =============================================================================
% BOOK INFORMATION
% =============================================================================
\newcommand{\booktitle}{The Art of Something\\[0.3em]
  \Large A Comprehensive Guide to Mastery}
\newcommand{\bookauthor}{Your Name}
\newcommand{\booksubtitle}{First Edition}
\newcommand{\publisher}{Self-Published / Publisher Name}
\newcommand{\publishyear}{2025}

% =============================================================================
% DOCUMENT CONTENT
% =============================================================================
\begin{document}

% =============================================================================
% FRONT MATTER
% =============================================================================
\frontmatter
\pagestyle{plain}

% --- Half Title ---
\begin{titlepage}
  \centering
  \vspace*{0.3\textheight}
  {\headingfont\LARGE The Art of Something\par}
  \vfill
\end{titlepage}

% --- Title Page ---
\begin{titlepage}
  \centering
  \vspace*{0.15\textheight}
  
  {\headingfont\Huge\bfseries\booktitle\par}
  
  \vspace{2cm}
  
  {\Large\bookauthor\par}
  
  \vfill
  
  {\large\publisher\par}
  \vspace{0.5cm}
  {\large\publishyear\par}
\end{titlepage}

% --- Copyright Page ---
\thispagestyle{empty}
\null
\vfill
\begin{flushleft}
  \textbf{The Art of Something}\\
  \booksubtitle\\[1em]
  Copyright © \publishyear\ \bookauthor\\[1em]
  All rights reserved. No part of this publication may be reproduced,
  distributed, or transmitted in any form or by any means without the prior
  written permission of the publisher.\\[1em]
  Typeset in IBM Plex using LuaLaTeX.\\[1em]
  ISBN: 000-0-00-000000-0\\[1em]
  \publisher\\
  City, Country\\[1em]
  www.example.com
\end{flushleft}
\clearpage

% --- Dedication ---
\thispagestyle{empty}
\null
\vspace{0.2\textheight}
\begin{center}
  \itshape
  To those who seek to understand,\\
  and to those who helped along the way.
\end{center}
\vfill
\clearpage

% --- Table of Contents ---
\tableofcontents

% --- Preface ---
\chapter{Preface}

Welcome to \textit{The Art of Something}. This book represents years of
experience and research distilled into a practical guide.

The book is organized into three parts:

\textbf{Part I: Foundations} covers the essential concepts and terminology
you'll need to understand the rest of the book.

\textbf{Part II: Core Techniques} presents the main methodologies and
approaches in depth.

\textbf{Part III: Advanced Topics} explores specialized areas and cutting-edge
developments.

Each chapter includes practical examples, exercises, and references for further
study.

\section*{How to Use This Book}

If you're new to the subject, I recommend reading the chapters in order.
Experienced readers may prefer to skip to specific chapters of interest.

Code examples are available at: \url{https://github.com/example/book-code}

\section*{Acknowledgements}

I would like to thank everyone who contributed to this book...

\vspace{1cm}
\begin{flushright}
  \textit{\bookauthor}\\
  \textit{City, \publishyear}
\end{flushright}

% =============================================================================
% MAIN MATTER
% =============================================================================
\mainmatter
\pagestyle{book}
\onehalfspacing

% =============================================================================
\part{Foundations}
% =============================================================================

% =====================================
\chapter{Introduction}
\label{ch:intro}
% =====================================

\begin{chapterquote}{Socrates}{}
The beginning of wisdom is the definition of terms.
\end{chapterquote}

\lettrine{T}{his chapter} introduces the fundamental concepts that form the
foundation of our subject. Understanding these basics is essential before
moving on to more advanced topics.

\section{What is This Book About?}

This book explores the fascinating world of [subject]. Whether you're a
complete beginner or an experienced practitioner, you'll find valuable insights
and practical techniques.

\begin{notebox}[title=Key Concept]
A \concept{central concept} is the fundamental building block of our discipline.
Understanding this concept is crucial for everything that follows.
\end{notebox}

\section{Historical Context}

The field has evolved significantly over the past decades. Understanding this
history helps contextualize current approaches.

\subsection{Early Developments}

In the beginning, practitioners faced significant challenges...

\subsection{Modern Era}

Today, we benefit from advances in technology and methodology...

\section{Core Principles}

Several core principles guide effective practice:

\begin{enumerate}
  \item \textbf{Principle One.} Description and rationale.
  \item \textbf{Principle Two.} Description and rationale.
  \item \textbf{Principle Three.} Description and rationale.
\end{enumerate}

\section{Summary}

This chapter has introduced:
\begin{itemize}
  \item The scope and purpose of this book
  \item Historical context and evolution
  \item Core principles that guide our approach
\end{itemize}

The next chapter will dive deeper into [next topic].

% =====================================
\chapter{Fundamental Concepts}
\label{ch:fundamentals}
% =====================================

\begin{chapterquote}{Albert Einstein}{1933}
Make things as simple as possible, but not simpler.
\end{chapterquote}

\lettrine{B}{uilding} on the introduction, this chapter presents the essential
concepts in greater detail.

\section{Concept One}

The first fundamental concept is...

\begin{citedquote}{Leading Expert}{2020}
Understanding this concept transformed how practitioners approach the problem.
Early misconceptions led to years of confusion before the breakthrough insight.
\end{citedquote}

\subsection{Definition}

Formally, we define...

\subsection{Properties}

Key properties include:

\begin{table}[htbp]
  \centering
  \caption{Properties of Concept One}
  \label{tab:properties}
  \begin{tblr}{
    colspec = {lX},
    row{1}  = {font=\bfseries\sffamily, bg=gray!15},
    hlines  = {0.5pt, gray!50},
    rowsep  = 3pt,
  }
    Property     & Description \\
    Property A   & Description of property A \\
    Property B   & Description of property B \\
    Property C   & Description of property C \\
  \end{tblr}
\end{table}

\subsection{Examples}

Let's look at some concrete examples...

\section{Concept Two}

The second fundamental concept builds on the first...

\section{Putting It Together}

When we combine these concepts...

\section{Exercises}

\begin{enumerate}
  \item Exercise one: description of the task.
  \item Exercise two: description of the task.
  \item Exercise three (challenging): description of a more difficult task.
\end{enumerate}

% =============================================================================
\part{Core Techniques}
% =============================================================================

% =====================================
\chapter{Technique One}
\label{ch:technique1}
% =====================================

\begin{chapterquote}{Unknown}{}
The only way to learn a new technique is to practice it.
\end{chapterquote}

\lettrine{N}{ow} that we've established the foundations, let's explore the
first core technique in depth.

\section{Overview}

This technique addresses...

\section{Step-by-Step Guide}

\subsection{Step 1: Preparation}

Before starting, ensure you have...

\subsection{Step 2: Implementation}

The core implementation involves:

\begin{codeblock}[python]
class TechniqueOne:
    """Implementation of Technique One."""
    
    def __init__(self, config):
        self.config = config
        self.state = self._initialize()
    
    def _initialize(self):
        """Set up initial state."""
        return {}
    
    def execute(self, input_data):
        """
        Main execution method.
        
        Args:
            input_data: The data to process
            
        Returns:
            Processed result
        """
        # Step 1: Validate
        self._validate(input_data)
        
        # Step 2: Transform
        transformed = self._transform(input_data)
        
        # Step 3: Process
        result = self._process(transformed)
        
        return result
\end{codeblock}

\subsection{Step 3: Verification}

After implementation, verify your results by...

\section{Common Pitfalls}

\begin{warningbox}
Avoid these common mistakes:
\begin{itemize}
  \item Mistake one and how to avoid it
  \item Mistake two and the correct approach
  \item Mistake three and its consequences
\end{itemize}
\end{warningbox}

\section{Advanced Variations}

For more sophisticated applications...

% =====================================
\chapter{Technique Two}
\label{ch:technique2}
% =====================================

\lettrine{T}{he} second technique complements the first...

\section{When to Use This Technique}

Use this technique when...

\section{Detailed Walkthrough}

Let's work through a complete example...

% =============================================================================
\part{Advanced Topics}
% =============================================================================

% =====================================
\chapter{Advanced Topic One}
\label{ch:advanced1}
% =====================================

\lettrine{F}{or} readers who have mastered the core techniques, this chapter
explores advanced applications...

\section{Prerequisites}

Before proceeding, ensure familiarity with:
\begin{itemize}
  \item \Cref{ch:fundamentals}: Fundamental concepts
  \item \Cref{ch:technique1}: Technique One
  \item \Cref{ch:technique2}: Technique Two
\end{itemize}

\section{Advanced Concepts}

Building on the foundations...

% =============================================================================
% BACK MATTER
% =============================================================================
\backmatter

% --- Appendices ---
\startappendices

\chapter{Reference Tables}
\label{app:reference}

Quick reference tables for common operations...

\chapter{Glossary}
\label{app:glossary}

\begin{description}
  \item[Term A] Definition of Term A.
  \item[Term B] Definition of Term B.
  \item[Term C] Definition of Term C.
\end{description}

% --- Bibliography ---
\printbibliography[heading=bibintoc, title={Bibliography}]

% --- Index ---
% \printindex  % Uncomment if using makeindex

% --- Colophon ---
\cleardoublepage
\thispagestyle{empty}
\null
\vfill
\begin{center}
  \small
  This book was typeset using LuaLaTeX with the \texttt{modern-doc} style.\\
  Body text is set in IBM Plex Serif, headings in IBM Plex Sans,\\
  and code in IBM Plex Mono.\\[1em]
  First printing, \publishyear
\end{center}

\end{document}
