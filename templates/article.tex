% =============================================================================
% ARTICLE TEMPLATE — Modern LuaLaTeX Article
% =============================================================================
% Compile with: lualatex --shell-escape article.tex
%               biber article
%               lualatex --shell-escape article.tex
% =============================================================================

% --- PDF 2.0 and metadata (must be before \documentclass) ---
\DocumentMetadata{
  pdfversion = 2.0,
  lang       = en-US,
}

% Suppress expected warning from latex-lab tagging (modifies \@footnotemark)
% This warning is harmless - latex-lab modifies footnotes for PDF accessibility
\RequirePackage{silence}
\WarningFilter{latex}{Command \@footnotemark}

\documentclass[11pt, a4paper, oneside]{scrartcl}

% =============================================================================
% PREAMBLE
% =============================================================================
\usepackage[article, citestyle=numeric]{moderndoc}

% --- Bibliography file ---
\addbibresource{references.bib}

% --- Document metadata ---
\hypersetup{
  pdftitle    = {Your Article Title},
  pdfauthor   = {Your Name},
  pdfsubject  = {Subject Area},
  pdfkeywords = {keyword1, keyword2, keyword3},
}

% --- Header content (plain text for running headers) ---
\shorttitle{Your Article Title}
\shortauthor{First Author \& Second Author}

% =============================================================================
% DOCUMENT CONTENT
% =============================================================================
\begin{document}

% --- Title ---
\title{Your Article Title}
\subtitle{A Subtitle if Needed}
\author{%
  First Author\thanks{Affiliation, email@example.com} \and
  Second Author\thanks{Another Affiliation, email2@example.com}
}
\date{\today}

\maketitle

% --- Abstract ---
\begin{abstract}
This is the abstract of your article. It should provide a concise summary of the
research question, methodology, key findings, and implications. Keep it to
150--300 words for most journals.

\begin{keywords}
typography, LaTeX, document preparation, academic writing, LuaLaTeX
\end{keywords}
\end{abstract}

% =============================================================================
\section{Introduction}
% =============================================================================

This template demonstrates the capabilities of the \texttt{moderndoc} style
package for creating professional academic articles using LuaLaTeX.

The IBM Plex font family provides a clean, modern aesthetic with excellent
readability. Notice how the serif text works harmoniously with the sans-serif
headings.

\subsection{Key Features}

\begin{itemize}
  \item Modern PDF 2.0 output with proper metadata
  \item IBM Plex font family with Japanese support
  \item Professional typography via \texttt{microtype}
  \item Modern bibliography with \texttt{biblatex}
  \item Code highlighting with \texttt{minted}
  \item Styled quote boxes with \texttt{tcolorbox}
  \item Modern tables with \texttt{tabularray}
\end{itemize}

% =============================================================================
\section{Typography Examples}
% =============================================================================

\subsection{Text Formatting}

Regular text flows naturally with \textbf{bold emphasis} and \textit{italic
text}. Small caps work well for \textsc{Acronyms} and \textsc{Proper Names}.

For Japanese terms, use the \verb|\jpterm| command: \jpterm{改善}{kaizen}
means continuous improvement. You can also insert Japanese text directly:
\jp{日本語テキスト}.

French text is also supported: \textfrench{Voici un exemple de texte en
français avec des guillemets « français »}.

\subsection{Quotations}

Use the \texttt{paperquote} environment for extracted quotes from papers:

\begin{paperquote}
The fundamental principles of good typography are legibility, readability, and
aesthetic appeal. These three elements must work together harmoniously to create
an effective document design.
\end{paperquote}

For attributed quotes, use \texttt{attributedquote}:

\begin{attributedquote}{Robert Bringhurst}
Typography exists to honor content. Good typography reveals rather than
conceals the message.
\end{attributedquote}

For quotes with author and year (proper citation style), use \texttt{citedquote}:

\begin{citedquote}{Robert Bringhurst}{1992}
Typography is the craft of endowing human language with a durable visual form,
and thus with an independent existence.
\end{citedquote}

For quick inline quotes, use the \verb|\q{}| command: \q{This is an inline quote}
with proper quotation marks that adapt to the document language.

% =============================================================================
\section{Code Examples}
% =============================================================================

Inline code looks like this: \mintinline{python}{def hello(): print("Hello")}.

For code blocks, use the \texttt{minted} environment:

\begin{minted}{python}
def fibonacci(n: int) -> int:
    """Calculate the nth Fibonacci number."""
    if n <= 1:
        return n
    return fibonacci(n - 1) + fibonacci(n - 2)

# Example usage
for i in range(10):
    print(f"F({i}) = {fibonacci(i)}")
\end{minted}

Multiple languages are supported:

\begin{minted}{javascript}
// Modern JavaScript with async/await
async function fetchData(url) {
  const response = await fetch(url);
  const data = await response.json();
  return data;
}
\end{minted}

% =============================================================================
\section{Tables}
% =============================================================================

Modern tables using \texttt{tabularray}:

\begin{table}[htbp]
  \centering
  \caption{Comparison of Document Classes}
  \label{tab:classes}
  \begin{tblr}{
    colspec = {lXcc},
    row{1}  = {font=\bfseries\sffamily, bg=gray!15},
    hlines  = {0.5pt, gray!50},
    rowsep  = 3pt,
  }
    Class     & Description                              & PDF 2.0 & Unicode \\
    scrartcl  & KOMA-Script article class               & Yes     & Yes     \\
    scrbook   & KOMA-Script book class                  & Yes     & Yes     \\
    scrreprt  & KOMA-Script report class                & Yes     & Yes     \\
    article   & Standard LaTeX article                  & Yes     & Limited \\
  \end{tblr}
\end{table}

% =============================================================================
\section{Boxes and Notes}
% =============================================================================

Use \texttt{notebox} for important information:

\begin{notebox}
This is an important note that readers should pay attention to. It stands out
from the main text with its colored border and background.
\end{notebox}

Use \texttt{warningbox} for warnings:

\begin{warningbox}
Be careful when using \texttt{--shell-escape} with untrusted documents, as it
allows LaTeX to execute arbitrary shell commands.
\end{warningbox}

% =============================================================================
\section{Mathematics}
% =============================================================================

Mathematics is rendered using STIX Two Math or Latin Modern Math:

The famous Euler's identity is: $e^{i\pi} + 1 = 0$.

For display equations:

\begin{equation}
  \int_{-\infty}^{\infty} e^{-x^2} \, dx = \sqrt{\pi}
  \label{eq:gaussian}
\end{equation}

As shown in \cref{eq:gaussian}, the Gaussian integral evaluates to $\sqrt{\pi}$.

% =============================================================================
\section{Cross-References and Citations}
% =============================================================================

The \texttt{cleveref} package enables smart cross-references. Refer to
\cref{tab:classes} for the comparison, or see \cref{sec:conclusion} for
concluding remarks.

Citations work seamlessly with biblatex. For example, you might cite a seminal
work\autocite{knuth1984texbook} or multiple sources\autocites{lamport1994latex}{mittelbach2004companion}.

% =============================================================================
\section{Semantic Markup}
% =============================================================================

The \texttt{moderndoc} package provides semantic markup commands:

\begin{itemize}
  \item \foreign{Foreign terms} for non-English words: \verb|\foreign{bon appétit}|
  \item \term{Technical terms} for terminology: \verb|\term{kerning}|
  \item \acronym{UNESCO} for acronyms: \verb|\acronym{UNESCO}|
  \item \bookref{The Elements of Typographic Style} for book titles
  \item \software{LuaTeX} for software names: \verb|\software{LuaTeX}|
  \item \email{example@email.com} for email addresses
\end{itemize}

Mathematical operators like $\argmax$ and $\argmin$ are predefined, along with
$\abs{x}$, $\norm{v}$, and $\set{a, b, c}$.

% =============================================================================
\section{Conclusion}
\label{sec:conclusion}
% =============================================================================

This template provides a solid foundation for academic articles with modern
typography and PDF features. The modular design allows easy customization while
maintaining professional appearance.

% =============================================================================
% BIBLIOGRAPHY
% =============================================================================
\printbibliography

\end{document}
