\DocumentMetadata{
  pdfversion=2.0,
  pdfstandard=a-4,
  lang=en-US,
}

\documentclass[11pt,a4paper,twoside,open=right]{scrbook}
\usepackage[
  doctype=book,
  language=en-US,
  font=plex,
  citestyle=authortitle,
  biblatex=true,
  bcor=8mm
]{moderndoc}

\addbibresource{../references.bib}

\hypersetup{
  pdftitle    = {Les Misérables},
  pdfauthor   = {Victor Hugo},
  pdfsubject  = {A novel about the struggles of ex-convict Jean Valjean},
}

\begin{document}

\title{Les Misérables}
\author{Victor Hugo\\[0.5em]
  \normalsize Translated by Isabel Florence Hapgood}
\date{1862}

\frontmatter
\maketitle

\chapter*{Preface}
\addcontentsline{toc}{chapter}{Preface}

So long as there shall exist, by virtue of law and custom, decrees of damnation
pronounced by society, artificially creating hells amid the civilization of
earth, and adding the element of human fate to divine destiny; so long as the
three great problems of the century---the degradation of man through pauperism,
the corruption of woman through hunger, the crippling of children through lack
of light---are unsolved; so long as social asphyxia is possible in any part of
the world;---in other words, and with a still wider significance, so long as
ignorance and poverty exist on earth, books of the nature of \emph{Les
Misérables} cannot fail to be of use.

\begin{flushright}
\caps{Hauteville House}, 1862.
\end{flushright}

\tableofcontents

\mainmatter

\part{Fantine}

\chapter{M.\ Myriel}

\dropcap{I}{n} 1815, M.\ Charles-François-Bienvenu Myriel was Bishop of D---.
He was an old man of about seventy-five years of age; he had occupied the see
of D--- since 1806.

Although this detail has no connection whatever with the real substance of what
we are about to relate, it will not be superfluous, if merely for the sake of
exactness in all points, to mention here the various rumors and remarks which
had been in circulation about him from the very moment when he arrived in the
diocese. True or false, that which is said of men often occupies as important a
place in their lives, and above all in their destinies, as that which they do.
M.\ Myriel was the son of a councillor of the Parliament of Aix; hence he
belonged to the nobility of the bar.

The Revolution came; events succeeded each other with precipitation; the
parliamentary families, decimated, pursued, hunted down, were dispersed. M.\
Charles Myriel emigrated to Italy at the very beginning of the Revolution.
There his wife died of a malady of the chest, from which she had long suffered.
He had no children.

In 1804, M.\ Myriel was the Curé of B---. He was already advanced in years, and
lived in a very retired manner.

About the epoch of the coronation, some petty affair connected with his
curacy---just what, is not precisely known---took him to Paris. Among other
powerful persons to whom he went to solicit aid for his parishioners was M.\ le
Cardinal Fesch. One day, when the Emperor had come to visit his uncle, the
worthy Curé, who was waiting in the anteroom, found himself present when His
Majesty passed. Napoleon, on finding himself observed with a certain curiosity
by this old man, turned round and said abruptly:

\begin{quote}
``Who is this good man who is staring at me?''

``Sire,'' said M.\ Myriel, ``you are looking at a good man, and I at a great
man. Each of us can profit by it.''
\end{quote}

That very evening, the Emperor asked the Cardinal the name of the Curé, and
some time afterwards M.\ Myriel was utterly astonished to learn that he had
been appointed Bishop of D---.

\chapter{M.\ Myriel Becomes M.\ Welcome}

\dropcap{T}{he} episcopal palace of D--- adjoins the hospital.

The episcopal palace was a huge and beautiful house, built of stone at the
beginning of the last century by M.\ Henri Puget, Doctor of Theology of the
Faculty of Paris, Abbé of Simore, who had been Bishop of D--- in 1712. This
palace was a genuine seignorial residence. Everything about it had a grand
air,---the apartments of the Bishop, the drawing-rooms, the chambers, the
principal courtyard, which was very large, with walks encircling it under
arcades in the old Florentine fashion, and gardens planted with magnificent
trees.

The hospital was a low and narrow building of a single story, with a small
garden.

Three days after his arrival, the Bishop visited the hospital. The visit ended,
he had the director requested to be so good as to come to his house.

``Monsieur the director of the hospital,'' said he to him, ``how many sick
people have you at the present moment?''

``Twenty-six, Monseigneur.''

``That was the number which I counted,'' said the Bishop.

The Bishop remained silent for a moment; then he turned abruptly to the
director of the hospital.

``Monsieur,'' said he, ``how many beds do you think this hall alone would
hold?''

``Monseigneur's dining-room?'' exclaimed the stupefied director.

``It would hold full twenty beds,'' said he, as though speaking to himself.
Then, raising his voice:

``Hold, Monsieur the director of the hospital, I will tell you something. There
is evidently a mistake here. There are thirty-six of you, in five or six small
rooms. There are three of us here, and we have room for sixty. There is some
mistake, I tell you; you have my house, and I have yours. Give me back my
house; you are at home here.''

On the following day the thirty-six patients were installed in the Bishop's
palace, and the Bishop was settled in the hospital.

\chapter{A Hard Bishopric for a Good Bishop}

\dropcap{T}{he} Bishop did not omit his pastoral visits because he had
converted his carriage into alms. The diocese of D--- is a fatiguing one. There
are very few plains and a great many mountains; hardly any roads, as we have
just seen; thirty-two curacies, forty-one vicarships, and two hundred and
eighty-five auxiliary chapels. To visit all these is quite a task.

The Bishop managed to do it. He went on foot when it was in the neighborhood,
in a tilted spring-cart when it was on the plain, and on a donkey in the
mountains. The two old women accompanied him. When the trip was too hard for
them, he went alone.

One day he arrived at Senez, which is an ancient episcopal city. He was mounted
on an ass. His purse, which was very dry at that moment, did not permit him any
other equipage. The mayor of the town came to receive him at the gate, and
watched him dismount from his ass, with scandalized eyes. Some of the citizens
were laughing around him.

``Monsieur the Mayor,'' said the Bishop, ``and Messieurs Citizens, I perceive
that I shock you; you think it very arrogant of a poor priest to ride an animal
which was used by Jesus Christ. I have done so from necessity, I assure you,
and not from vanity.''

\backmatter

\chapter*{About This Text}
\addcontentsline{toc}{chapter}{About This Text}

This excerpt from \emph{Les Misérables} is reproduced from the Project
Gutenberg edition (eBook \#135) \autocite{hugo1862miserables}, which is in the
public domain. The original French text was published in 1862, and this
English translation by Isabel Florence Hapgood was published in 1887.

\begin{description}
  \item[Original Author] Victor Hugo (1802--1885)
  \item[Translator] Isabel Florence Hapgood (1850--1928)
  \item[Source] Project Gutenberg, \url{https://www.gutenberg.org/ebooks/135}
  \item[License] Public Domain
\end{description}

\printbibliography

\end{document}
