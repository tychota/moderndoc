\DocumentMetadata{
  pdfversion=2.0,
  pdfstandard=a-4,
  lang=en-US,
}

\documentclass[thesis]{modern-doc}

\addbibresource{../references.bib}

\hypersetup{
  pdftitle    = {Aesthetic-Aware Document Layout via Deep Learning},
  pdfauthor   = {Jane Scholar},
}

\begin{document}

\frontmatter
\pagestyle{plain}

\begin{titlepage}
  \centering
  \vspace*{2cm}
  {\headingfont\LARGE University of Typography\par}
  \vspace{0.5cm}
  {\headingfont\large Department of Computer Science\par}
  
  \vspace{3cm}
  
  {\headingfont\Huge\bfseries Aesthetic-Aware Document Layout via Deep Learning\par}
  
  \vspace{2cm}
  
  {\Large Master's Thesis\par}
  
  \vspace{2cm}
  
  {\Large Jane Scholar\par}
  
  \vfill
  
  {\large Supervised by:\par}
  {\large Prof. Donald Knuth\par}
  
  \vspace{1cm}
  
  {\large December 2025\par}
\end{titlepage}

\begin{thesisabstract}
This thesis explores the integration of typographic design principles into
automated document layout systems. While deep learning has achieved state-of-the-art
results in document understanding, generating layouts that are both functional
and aesthetically pleasing remains a challenge.

We propose a novel neural architecture that explicitly models \q{white space}
and alignment. Our results, validated through user studies and quantitative
metrics, demonstrate a significant improvement over baseline generative models.
\end{thesisabstract}

\tableofcontents
\listoffigures
\listoftables

\mainmatter
\pagestyle{scrheadings}

\chapter{Introduction}

\section{Motivation}

The proliferation of digital documents has made professional-grade typography
more accessible than ever. However, the quality of automated layout generation
often falls short of human-designed counterparts.

\section{Problem Statement}

Current layout algorithms focus primarily on content density and fitting
constraints, often neglecting the subtle interplay of fonts, margins, and
visual hierarchy that defines good typography.

\chapter{Literature Review}

\section{Historical Perspective}

As Bringhurst argues, typography is the craft of endowing human language with a
durable visual form\autocite{bringhurst2004elements}. From the mechanical
presses of Gutenberg to the digital precision of \TeX\autocite{carroll1865alice}, the
fundamental goal remains legibility.

\section{AI in Document Analysis}

Recent work by \textcite{yang2024aesthetics} suggests that AI models can be
trained to recognize aesthetic qualities, though their confidence does not always
align with human perception.

\chapter{Methodology}

\section{Neural Layout Model}

We utilize a Transformer-based architecture to model the dependencies between
various document elements.

\begin{codeblock}[python]
class LayoutTransformer(nn.Module):
    def __init__(self, vocab_size, d_model):
        super().__init__()
        self.embedding = nn.Embedding(vocab_size, d_model)
        self.transformer = nn.Transformer(d_model)
        
    def forward(self, x):
        x = self.embedding(x)
        return self.transformer(x)
\end{codeblock}

\chapter{Evaluation}

\section{Datasets}

We evaluate our model on the DocBank and PubLayNet datasets, which contain
millions of annotated document pages.

\chapter{Conclusion}

Our work bridges the gap between traditional design theory and modern
computational methods. Future work will focus on interactive layout assistants
that collaborate with human designers.

\backmatter
\printbibliography

\end{document}