\DocumentMetadata{
  pdfversion=2.0,
  pdfstandard=a-4,
  lang=en-US,
}

\documentclass[12pt,a4paper,twoside,open=right]{scrbook}
\usepackage[
  doctype=thesis,
  language=en-US,
  font=plex,
  citestyle=authoryear,
  biblatex=true,
  bcor=10mm
]{moderndoc}

\addbibresource{../references.bib}

\hypersetup{
  pdftitle    = {The Empiricist Theory of Ideas},
  pdfauthor   = {Jane Scholar},
}

\begin{document}

\frontmatter
\pagestyle{plain}

\begin{titlepage}
  \centering
  \vspace*{2cm}
  {\headingfont\LARGE Open University of Philosophy\par}
  \vspace{0.5cm}
  {\headingfont\large Department of Epistemology\par}

  \vspace{3cm}

  {\headingfont\Huge\bfseries The Empiricist Theory of Ideas:\\
    A Study of Hume's Treatise\par}

  \vspace{2cm}

  {\Large Master's Thesis\par}

  \vspace{2cm}

  {\Large Jane Scholar\par}

  \vfill

  {\large Supervised by:\par}
  {\large Prof.\ Maria Epistemology\par}

  \vspace{1cm}

  {\large December 2025\par}

  \vspace{1cm}

  {\small Based on public domain texts from Project Gutenberg\par}
\end{titlepage}

\begin{thesisabstract}
This thesis examines David Hume's empiricist theory of ideas as presented in
\emph{A Treatise of Human Nature}~\autocite{hume1739treatise}. Hume's
fundamental distinction between impressions and ideas, and his account of how
the mind associates ideas through resemblance, contiguity, and causation,
remain central to contemporary debates in epistemology and philosophy of mind.

We analyze Hume's arguments for the primacy of sense experience in the
formation of knowledge, his distinction between simple and complex perceptions,
and his influential account of memory and imagination. Building on these
foundations, we explore the implications for contemporary theories of mental
representation and cognitive architecture.

This study demonstrates that Hume's empiricist framework, while formulated in
the eighteenth century, continues to offer insights relevant to modern
cognitive science and philosophy of mind.
\end{thesisabstract}

\tableofcontents

\mainmatter
\pagestyle{scrheadings}

\chapter{Introduction}

\section{Motivation}

The question of how human beings acquire knowledge has occupied philosophers
since antiquity. David Hume's \emph{A Treatise of Human
Nature}~\autocite{hume1739treatise}, first published in 1739, represents one of
the most systematic attempts to ground human understanding in experience.

Hume's empiricist project seeks to explain all mental contents---our ideas,
beliefs, and knowledge---as ultimately derived from sensory impressions. As he
writes in the opening of the Treatise, all the perceptions of the human mind
resolve themselves into two distinct kinds, which he calls impressions and
ideas.

\section{Problem Statement}

This thesis addresses several key questions arising from Hume's theory:
\begin{enumerate}
  \item What is the precise relationship between impressions and ideas?
  \item How does the mind form complex ideas from simple components?
  \item What principles govern the association of ideas?
  \item How do memory and imagination differ in their operations?
\end{enumerate}

\section{Contributions}

This thesis makes the following contributions:
\begin{itemize}
  \item A systematic analysis of Hume's impression-idea distinction
  \item An examination of the principles of mental association
  \item A comparison with contemporary theories of mental representation
  \item An assessment of Hume's relevance to cognitive science
\end{itemize}

\chapter{Impressions and Ideas}

\section{The Fundamental Distinction}

All the perceptions of the human mind resolve themselves into two distinct
kinds, which Hume calls \textsc{impressions} and \textsc{ideas}. The difference
betwixt these consists in the degrees of force and liveliness, with which they
strike upon the mind, and make their way into our thought or consciousness.

Those perceptions which enter with most force and violence, we may name
impressions: and under this name Hume comprehends all our sensations, passions
and emotions, as they make their first appearance in the soul. By ideas he
means the faint images of these in thinking and reasoning.

\subsection{Force and Vivacity}

Every one of himself will readily perceive the difference betwixt feeling and
thinking. The common degrees of these are easily distinguished; though it is
not impossible but in particular instances they may very nearly approach to
each other. Thus in sleep, in a fever, in madness, or in any very violent
emotions of soul, our ideas may approach to our impressions.

\subsection{Terminological Innovation}

\begin{notebox}
Hume makes use of the terms \emph{impression} and \emph{idea} in a sense
different from what was usual in his time. He claims to restore the word
``idea'' to its original sense, from which Locke had perverted it in making it
stand for all our perceptions.
\end{notebox}

\section{Simple and Complex Perceptions}

There is another division of our perceptions, which extends itself both to our
impressions and ideas. This division is into \textsc{simple} and
\textsc{complex}. Simple perceptions or impressions and ideas are such as admit
of no distinction nor separation. The complex are the contrary to these, and
may be distinguished into parts.

Though a particular colour, taste, and smell, are qualities all united together
in this apple, it is easy to perceive they are not the same, but are at least
distinguishable from each other.

\chapter{The Association of Ideas}

\section{The Need for Associating Principles}

As all simple ideas may be separated by the imagination, and may be united
again in what form it pleases, nothing would be more unaccountable than the
operations of that faculty, were it not guided by some universal principles,
which render it, in some measure, uniform with itself in all times and places.

Were ideas entirely loose and unconnected, chance alone would join them; and it
is impossible the same simple ideas should fall regularly into complex ones (as
they commonly do) without some bond of union among them, some associating
quality, by which one idea naturally introduces another.

\section{The Three Principles of Association}

The qualities from which this association arises, and by which the mind is
conveyed from one idea to another, are three:

\begin{enumerate}
  \item \textbf{Resemblance}: In the course of our thinking, and in the
    constant revolution of our ideas, our imagination runs easily from one idea
    to any other that resembles it.
  \item \textbf{Contiguity}: As the senses, in changing their objects, are
    necessitated to change them regularly, and take them as they lie contiguous
    to each other, the imagination must by long custom acquire the same method
    of thinking.
  \item \textbf{Cause and Effect}: There is no relation which produces a
    stronger connexion in the fancy, and makes one idea more readily recall
    another, than the relation of cause and effect betwixt their objects.
\end{enumerate}

\section{The Extent of These Relations}

Two objects are connected together in the imagination, not only when the one is
immediately resembling, contiguous to, or the cause of the other, but also when
there is interposed betwixt them a third object, which bears to both of them
any of these relations. This may be carried on to a great length; though at the
same time we may observe, that each remove considerably weakens the relation.

Of the three relations above-mentioned, causation is the most extensive. Two
objects may be considered as placed in this relation, as well when one is the
cause of any of the actions or motions of the other, as when the former is the
cause of the existence of the latter.

\chapter{Memory and Imagination}

\section{The Distinction}

It is evident at first sight that the ideas of the memory are much more lively
and strong than those of the imagination, and that the former faculty paints
its objects in more distinct colours than any which are employed by the latter.
When we remember any past event, the idea of it flows in upon the mind in a
forcible manner; whereas in the imagination the perception is faint and
languid, and cannot without difficulty be preserved by the mind steady and
uniform for any considerable time.

\section{Order and Form}

Though neither the ideas of the memory nor imagination can make their
appearance in the mind unless their correspondent impressions have gone before
to prepare the way for them, yet the imagination is not restrained to the same
order and form with the original impressions; while the memory is in a manner
tied down in that respect, without any power of variation.

It is evident that the memory preserves the original form in which its objects
were presented, and that wherever we depart from it in recollecting anything,
it proceeds from some defect or imperfection in that faculty.

\section{The Liberty of Imagination}

The same evidence follows us in our second principle, of the liberty of the
imagination to transpose and change its ideas. The fables we meet with in poems
and romances put this entirely out of the question. Nature there is totally
confounded, and nothing mentioned but winged horses, fiery dragons, and
monstrous giants.

\chapter{Conclusion}

\section{Summary}

Hume's empiricist theory provides a systematic account of the origin and
association of our ideas. By grounding all mental contents in sensory
impressions, and explaining the formation of complex ideas through the
principles of resemblance, contiguity, and causation, Hume offers a naturalistic
framework for understanding human cognition.

\section{Contemporary Relevance}

The principles Hume identified remain relevant to contemporary cognitive
science:
\begin{itemize}
  \item Associative learning models in psychology
  \item Connectionist theories of mental representation
  \item Theories of memory consolidation and retrieval
  \item Philosophical debates about the nature of mental content
\end{itemize}

\backmatter

\chapter*{Acknowledgments}
\addcontentsline{toc}{chapter}{Acknowledgments}

I thank my supervisor, Prof.\ Maria Epistemology, for her guidance throughout
this research. The primary texts analyzed in this thesis are drawn from David
Hume's \emph{A Treatise of Human Nature}, available through Project Gutenberg
(eBook \#4705) in the public domain.

\vspace{1em}
\noindent\textit{Source: Project Gutenberg eBook \#4705. Public Domain.}

\printbibliography

\end{document}
