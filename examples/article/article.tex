\DocumentMetadata{
  pdfversion=2.0,
  pdfstandard=a-4,
  lang=en-US,
}

\documentclass[article]{modern-doc}

\addbibresource{../references.bib}

\hypersetup{
  pdftitle    = {Use of LLMs in preparing accessible scientific papers},
  pdfauthor   = {Allison Doami \and Christine James \and Dan Lu \and \\
  Lia Prins \and Annette Torrence \and Boris Veytsman},
}

\shorttitle{LLMs for Accessible Papers}
\shortauthor{A. Doami et al.}

\begin{document}

\title{Use of LLMs in preparing accessible scientific papers}
\author{
  Allison Doami \and Christine James \and Dan Lu \and \\
  Lia Prins \and Annette Torrence \and Boris Veytsman
}
\date{May 7, 2025}

\maketitle

\begin{abstract}
Making scientific papers accessible may require reprocessing old papers to
create output compliant with accessibility standards. An important step there is
to convert the visual formatting to the logical one. In this report we describe
our attempt at zero shot conversion of arXiv papers. Our results are mixed: while
it is possible to do conversion, the reliability is not too good. We discuss
alternative approaches to this problem.

\begin{keywords}
accessibility, LLM, LaTeX, arXiv, document conversion
\end{keywords}
\end{abstract}

\section{Introduction}

The scientific community has produced a vast repository of knowledge in the form
of research papers, predominantly distributed as PDF files. However, a
significant portion of this legacy content is not accessible to individuals
with disabilities, failing to meet modern standards such as WCAG or PDF/UA.

Making scientific papers accessible is not merely a legal or ethical obligation
but a necessity for the broad dissemination of scientific knowledge. For older
papers, where the original source (e.g., \\TeX or \\LaTeX) may be lost or
incompatible with modern tools, the challenge is even greater.

\section{Logical vs. Visual Formatting}

A key hurdle in document accessibility is the distinction between visual and
logical formatting. While humans can easily infer the structure of a document
(headings, lists, equations) from its visual layout, assistive technologies
require explicit logical markup.

\begin{notebox}
\term{Visual formatting} refers to how the text looks on the page (bold, large
fonts, spacing). \term{Logical formatting} defines what the text is (a section
heading, a figure caption, a mathematical variable).
\end{notebox}

\section{LLMs for Zero-Shot Conversion}

Large Language Models (LLMs) have shown remarkable capabilities in understanding
and transforming text. We explored the possibility of using LLMs to convert
visually formatted text extracted from PDFs back into structured \\LaTeX or
HTML.

\begin{codeblock}[python]
def convert_to_accessible(text):
    # Pseudocode for LLM-based conversion
    prompt = f"Convert the following visual text to logical LaTeX:\n{text}"
    accessible_tex = llm.generate(prompt)
    return accessible_tex
\end{codeblock}

\section{Results and Observations}

Our experiments with zero-shot conversion yielded mixed results. While the LLMs
were often able to identify major structural elements like section titles, they
frequently struggled with complex mathematical formulas and tables.

As noted in \\textcite{yang2024aesthetics}, the aesthetic quality of a document
can influence how its information is perceived, but for AI models, the visual
clues are often secondary to the textual patterns.

\section{Conclusion}

While LLMs offer a promising path for automating the creation of accessible
scientific documents, the current reliability is insufficient for fully
automated pipelines. Further research into fine-tuned models or hybrid
approaches combining LLMs with traditional OCR/layout analysis is needed.

\printbibliography

\end{document}
