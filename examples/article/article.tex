\DocumentMetadata{
  pdfversion=2.0,
  pdfstandard=a-4,
  lang=en-US,
}

\documentclass[11pt,a4paper]{scrartcl}
\usepackage[
  doctype=article,
  language=en-US,
  font=plex,
  citestyle=numeric,
  biblatex=true
]{moderndoc}

\addbibresource{../references.bib}

\hypersetup{
  pdftitle    = {Of Ideas, Their Origin and Association},
  pdfauthor   = {David Hume},
}

\shorttitle{Of Ideas}
\shortauthor{Hume}

\begin{document}

\title{Of Ideas, Their Origin and Association:\\
  \large Excerpts from A Treatise of Human Nature}
\author{David Hume}
\date{1739}

\maketitle

\begin{abstract}
This article presents key excerpts from David Hume's \emph{A Treatise of Human
Nature}~\autocite{hume1739treatise}, first published in 1739. Hume's empiricist
philosophy examines the foundations of human understanding, distinguishing
between impressions and ideas, and exploring the principles by which the mind
associates one idea with another. These selections from Book~I demonstrate
Hume's systematic approach to epistemology and his influence on subsequent
philosophical inquiry.

\begin{keywords}
empiricism, impressions, ideas, association, causation, philosophy
\end{keywords}
\end{abstract}

\section{Of the Origin of Our Ideas}

All the perceptions of the human mind resolve themselves into two distinct
kinds, which I shall call \textsc{impressions} and \textsc{ideas}. The
difference betwixt these consists in the degrees of force and liveliness, with
which they strike upon the mind, and make their way into our thought or
consciousness. Those perceptions, which enter with most force and violence, we
may name impressions: and under this name I comprehend all our sensations,
passions and emotions, as they make their first appearance in the soul. By
ideas I mean the faint images of these in thinking and reasoning; such as, for
instance, are all the perceptions excited by the present discourse, excepting
only those which arise from the sight and touch, and excepting the immediate
pleasure or uneasiness it may occasion.

I believe it will not be very necessary to employ many words in explaining this
distinction. Every one of himself will readily perceive the difference betwixt
feeling and thinking. The common degrees of these are easily distinguished;
though it is not impossible but in particular instances they may very nearly
approach to each other. Thus in sleep, in a fever, in madness, or in any very
violent emotions of soul, our ideas may approach to our impressions. As on the
other hand it sometimes happens, that our impressions are so faint and low,
that we cannot distinguish them from our ideas.

\begin{notebox}
Hume here makes use of the terms \emph{impression} and \emph{idea} in a sense
different from what was usual in his time, restoring the word ``idea'' to its
original sense from which Locke had perverted it.
\end{notebox}

\section{Simple and Complex Perceptions}

There is another division of our perceptions, which it will be convenient to
observe, and which extends itself both to our impressions and ideas. This
division is into \textsc{simple} and \textsc{complex}. Simple perceptions or
impressions and ideas are such as admit of no distinction nor separation. The
complex are the contrary to these, and may be distinguished into parts. Though
a particular colour, taste, and smell, are qualities all united together in
this apple, it is easy to perceive they are not the same, but are at least
distinguishable from each other.

Having by these divisions given an order and arrangement to our objects, we may
now apply ourselves to consider with the more accuracy their qualities and
relations. The first circumstance, that strikes my eye, is the great
resemblance betwixt our impressions and ideas in every other particular, except
their degree of force and vivacity. The one seem to be in a manner the
reflexion of the other; so that all the perceptions of the mind are double, and
appear both as impressions and ideas.

\section{Of the Association of Ideas}

The qualities, from which this association arises, and by which the mind is
after this manner conveyed from one idea to another, are three, viz.\
\textsc{resemblance}, \textsc{contiguity} in time or place, and \textsc{cause
and effect}.

\begin{enumerate}
  \item \textbf{Resemblance}: In the course of our thinking, and in the
    constant revolution of our ideas, our imagination runs easily from one idea
    to any other that resembles it.
  \item \textbf{Contiguity}: As the senses, in changing their objects, are
    necessitated to change them regularly, and take them as they lie contiguous
    to each other, the imagination must by long custom acquire the same method
    of thinking.
  \item \textbf{Cause and Effect}: There is no relation which produces a
    stronger connexion in the fancy, and makes one idea more readily recall
    another, than the relation of cause and effect betwixt their objects.
\end{enumerate}

That we may understand the full extent of these relations, we must consider,
that two objects are connected together in the imagination, not only when the
one is immediately resembling, contiguous to, or the cause of the other, but
also when there is interposed betwixt them a third object, which bears to both
of them any of these relations. This may be carried on to a great length;
though at the same time we may observe, that each remove considerably weakens
the relation.

\section{Conclusion}

Of the three relations above-mentioned, causation is the most extensive. Two
objects may be considered as placed in this relation, as well when one is the
cause of any of the actions or motions of the other, as when the former is the
cause of the existence of the latter. These are therefore the principles of
union or cohesion among our simple ideas, and in the imagination supply the
place of that inseparable connexion, by which they are united in our memory.

\vspace{1em}
\noindent\textit{Source: Project Gutenberg eBook \#4705. Public Domain.}

\printbibliography

\end{document}
